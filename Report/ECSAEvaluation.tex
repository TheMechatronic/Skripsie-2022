% Title: Report LaTex File: ECSA Evaluation - Filled in
% Auther: DC Eksteen
% Student Number: 22623906
% Contact: 22623906@sun.ac.za
% Date: 2022/09/14
% Version: 2.0

\makeatletter
\newcommand{\ECSApage}{
	\clearpage
	\vspace*{15mm - 1in-\voffset-\topmargin-\headheight-\headsep-\topskip}
	\thispagestyle{plain}%
	\centering
	\phantomsection
	\addcontentsline{toc}{chapter}{ECSA Exit Level Outcome Evaluation}%
	\begin{USS@SetMargins}{25mm}{25mm}
		\centering\large\sffamily\bfseries\MakeUppercase{\DegreeNameLong: \\ECSA Exit Level Outcome Evaluation}
		\bigskip
		\setlength\LTleft{\leftmargin}%
		\setlength\LTright{\rightmargin}
		\renewcommand{\arraystretch}{1}
		\small
		\centering
		\begin{longtable}{@{\extracolsep{\fill}}| >{\raggedright}p{.85\textwidth} | >{\raggedright\noindent\arraybackslash}p{32mm} |}%
			\hline
			\multicolumn{2}{|>{\small\sffamily\bfseries\columncolor[gray]{.8}}c|}{\capitalisewords{ELO 1: Problem Solving}}                                                                                                                                                                                                \\
			\hline
			Demonstrate competence to identify, assess, formulate and solve convergent and divergent engineering problems creatively and innovatively.                       & Chapters:\newline 1 \& 3                                                                                                                    \\*
			\nobreakhline
			\multicolumn{2}
			{@{\hspace{\fill}} >{\small\normalfont\justifying}p{\textwidth} @{\hspace{\fill}}}{
			\begin{itemize}[leftmargin=*]
				\item The need for well-documented open-source software allowing connectivity with the Zwift application was identified. It was also determined that this software could be used to demonstrate the development of an inexpensive smart trainer. These needs are documented as project objectives in Chapter~1.
				\item The requirements for interaction with the Zwift application, as well as the developed smart-trainer are identified and refined into functional and performance requirements in Chapter~\ref{ch:concept}.
				\item Chapter~\ref{ch:concept} presents different concept solutions that were developed. These concept solutions are then evaluated against the target requirements and the best solution for the project objectives was then selected.
			\end{itemize}} \\
			\hline
			\multicolumn{2}{|>{\small\sffamily\bfseries\columncolor[gray]{.8}}c|}{\capitalisewords{ELO 2: Application of scientific and engineering knowledge}}                                                                                                                                                            \\
			\nobreakhline
			Demonstrate competence to apply knowledge of mathematics, basic science and engineering sciences from first principles to solve engineering problems.            &               Chapters:
			\newline \ref{ch:concept}, \ref{ch:hardware}, \ref{ch:electronics} \& \ref{ch:testing}                                                                                           \\*
			\nobreakhline
			\multicolumn{2}
			{@{\hspace{\fill}} >{\small\normalfont\justifying}p{\textwidth} @{\hspace{\fill}}}{
			\begin{itemize}[leftmargin=*]
				\item The development of the smart trainer includes a mathematical analysis of the operating conditions and requirements presented in Chapters~\ref{ch:concept} and \ref{ch:hardware}.
				\item The development of electronic components and sub-components based on mathematical and physical laws is presented in Chapter~\ref{ch:electronics}.
				\item The development and numerical analysis of an eddy-current brake based on the concept of electromagnetic induction is presented in Chapter~\ref{ch:hardware}.
				\item The fitting and verification of a surface plot to the experimentally determined torque characteristics of the developed trainer is presented in Chapter~\ref{ch:testing}.
			\end{itemize}
			}                                                                                                                                                                                                                                                                                                              \\
			\hline
			\multicolumn{2}{|>{\small\sffamily\bfseries\columncolor[gray]{.8}}c|}{\capitalisewords{
			ELO 3: Engineering Design}}                                                                                                                                                                                                                                                                                    \\
			\nobreakhline
			Demonstrate competence to perform creative, procedural and non-procedural design and synthesis of components, systems, engineering works, products or processes. & Chapters:\newline \ref{ch:concept}, \ref{ch:hardware}, \ref{ch:software}, \ref{ch:electronics} \& \ref{ch:testing}                                                                                                       \\
			\nobreakhline
			\multicolumn{2}
			{@{\hspace{\fill}} >{\small\normalfont\justifying}p{\textwidth} @{\hspace{\fill}}}{
			\begin{itemize}[leftmargin=*]
				\item The design process of two systems is demonstrated in this report. The software solution for interacting with Zwift is presented in Chapter~\ref{ch:software} and the hardware and electronics design in Chapters~\ref{ch:hardware}~and~\ref{ch:electronics}.
				\item The system constraints, boundaries and interfaces are considered and evaluated in Chapter~\ref{ch:concept} by following a systems engineering approach.
				\item The design process for the hardware, software and electronics is documented in Chapters~\ref{ch:hardware},~\ref{ch:software}~and~\ref{ch:electronics} respectively. The system integration is then tested and demonstrated in Chapter~\ref{ch:testing}.
			\end{itemize}
			}                                                                                                                                                                                                                                                                                                              \\
			\hline
			\multicolumn{2}{|>{\small\sffamily\bfseries\columncolor[gray]{.8}}c|}{\capitalisewords{ELO 5: Engineering methods, skills and tools, including Information Technology}}                                                                                                                                        \\
			\nobreakhline
			Demonstrate competence to use appropriate engineering methods, skills and tools, including those based on information technology.                                & Chapters:\newline\ref{ch:hardware}, \ref{ch:software}, \ref{ch:electronics} \& \ref{ch:testing}                                                                                                   \\
			\nobreakhline
			\multicolumn{2}
			{@{\hspace{\fill}} >{\small\normalfont\justifying}p{\textwidth} @{\hspace{\fill}}}{
			\begin{itemize}[leftmargin=*]
				\item A high level of knowledge of computer systems and electronics design is demonstrated in Chapters~\ref{ch:software}~and~\ref{ch:electronics}.
				\item Knowledge of data analysis, numerical methods and control systems is demonstrated in Chapters~\ref{ch:electronics}~and~\ref{ch:testing}, where fitting and validation of a model is performed and used in the control of the stepper motor with feedback from a pair of speed sensors.
				\item Knowledge of machine and mechanical design is demonstrated in Chapter~\ref{ch:hardware} where appropriate mechanical and mechatronic components were designed, selected and implemented.
			\end{itemize}
			}                                                                                                                                                                                                                                                                                                              \\
			\hline
			\multicolumn{2}{|>{\small\sffamily\bfseries\columncolor[gray]{.8}}c|}{\capitalisewords{ELO 6: Professional and technical communication}}                                                                                                                                                                       \\
			\nobreakhline
			Demonstrate competence to communicate effectively, both orally and in writing, with engineering audiences and the community at large.                            & Report\newline Presentation\newline Video                                    \\
			\nobreakhline
			\multicolumn{2}
			{@{\hspace{\fill}} >{\small\normalfont\justifying}p{\textwidth} @{\hspace{\fill}}}{
			\begin{itemize}[leftmargin=*]
				\item This report primarily aims to provide feedback on the technical aspects of the completed project, but also serve as reference material for future projects aimed at similar work. The report is thus technical in nature and does not serve as a step\nobreakdash-by\nobreakdash-step guide for reproducing the achieved results.
				\item The report has been presented as a free-standing document aimed at a professional audience with an academic background in engineering. The report will be supplemented by an oral presentation with a focus on the most important outcomes of the project, and finally, a non-technical video will serve to demonstrate the project and its outcome to an audience without a technical or engineering background.
			\end{itemize}
			}                                                                                                                                                                                                                                                                                                              \\
			\hline
			\multicolumn{2}{|>{\small\sffamily\bfseries\columncolor[gray]{.8}}c|}{\capitalisewords{ELO 8: Individual, Team and Multidisciplinary Working}}                                                                                                                                                                 \\
			\nobreakhline
			Demonstrate competence to work effectively as an individual, in teams and in multi-disciplinary environments.                                                    & Report\newline Project Proposal\newline Presentations                                                                                                             \\
			\nobreakhline
			\multicolumn{2}
			{@{\hspace{\fill}} >{\small\normalfont\justifying}p{\textwidth} @{\hspace{\fill}}}{
			\begin{itemize}[leftmargin=*]
				\item This project was completed as individual work in demonstrating and solving the project objectives.
				\item Collaboration with lab managers, external suppliers and workshop technicians was required to achieve the project objectives.
				\item During the progress of the project, bi-weekly progress presentations, a formal project proposal, a formal progress presentation and a first draft was delivered as required by the project brief document.
			\end{itemize}
			}                                                                                                                                                                                                                                                                                                              \\
			\hline
			\multicolumn{2}{|>{\small\sffamily\bfseries\columncolor[gray]{.8}}c|}{\capitalisewords{ELO 9: Independent Learning Ability}}                                                                                                                                                                                   \\
			\nobreakhline
			Demonstrate competence to engage in independent learning through well-developed learning skills.
			                                                                                                                                                                 & Chapter:\newline \ref{ch:software}                                                                                                        \\
			\nobreakhline
			\multicolumn{2}
			{@{\hspace{\fill}} >{\small\normalfont\justifying}p{\textwidth} @{\hspace{\fill}}}{
			\begin{itemize}[leftmargin=*]
				\item The use of sophisticated Bluetooth Low Energy specification sheets to implement Bluetooth communication protocols is demonstrated in Chapter~\ref{ch:software}.
				\item The use of the Linux operating system and Raspberry Pi system was studied and implemented as demonstrated in Chapter~\ref{ch:software}.
				\item The efficient use of the Python programming language was studied and implemented for use in this project. This included an investigation into common practice when creating open-source software that can easily be adapted for cross-platform implementation and is demonstrated in Chapter~\ref{ch:software}.
			\end{itemize}
			}
		\end{longtable}
	\end{USS@SetMargins}
	\clearpage
}
\makeatother


\makeatletter
\newcommand{\ECSApageInfo}{
	\clearpage
	\vspace*{15mm - 1in-\voffset-\topmargin-\headheight-\headsep-\topskip}
	\thispagestyle{plain}%
	\centering
	\phantomsection
	\addcontentsline{toc}{chapter}{ECSA Exit Level Outcome Evaluation}%
	\begin{USS@SetMargins}{25mm}{25mm}
		\centering\large\sffamily\bfseries\MakeUppercase{\DegreeNameLong: \\ECSA Exit Level Outcome Evaluation}
		\bigskip
		\setlength\LTleft{\leftmargin}%
		\setlength\LTright{\rightmargin}
		\renewcommand{\arraystretch}{1}
		\small
		\centering
		\begin{longtable}{@{\extracolsep{\fill}}| >{\raggedright}p{.85\textwidth} | >{\raggedright\noindent\arraybackslash}p{32mm} |}%
			\hline
			\multicolumn{2}{|>{\small\sffamily\bfseries\columncolor[gray]{.8}}c|}{\capitalisewords{ELO 1: Problem Solving}}                                                                                                                  \\
			\hline
			Demonstrate competence to identify, assess, formulate and solve convergent and divergent engineering problems creatively and innovatively.                       & \textbullet \space Section 1\newline \textbullet\space Poster \\*
			\nobreakhline
			\multicolumn{2}
			{@{\hspace{\fill}} >{\small\normalfont\justifying}p{\textwidth} @{\hspace{\fill}}}{\par The candidate must successfully complete an individual project. The project is
			assessed using a number of reports, oral presentations and poster as stipulated in the
			due dates section at the start of this document.
			The candidate applies in a number of varied instances, a systematic problem solving
			method including:
			\begin{enumerate}
				\item Analyses and defines the problem, identifies the criteria for an acceptable solution
				\item Identifies necessary information and applicable engineering and other knowledge and skills
				\item Generates and formulates possible approaches to solution of problem
				\item Models and analyses possible solution(s)
				\item Evaluates possible solutions and selects best solution
				\item Formulates and presents the solution in an appropriate form.
			\end{enumerate}}                                                                                                                                  \\
			\hline
			\multicolumn{2}{|>{\small\sffamily\bfseries\columncolor[gray]{.8}}c|}{\capitalisewords{ELO 2: Application of scientific and engineering knowledge}}                                                                              \\
			\nobreakhline
			Demonstrate competence to apply knowledge of mathematics, basic science and engineering sciences from first principles to solve engineering problems.            & \textbullet \space Section X                                  \\*
			\nobreakhline
			\multicolumn{2}
			{@{\hspace{\fill}} >{\small\normalfont\justifying}p{\textwidth} @{\hspace{\fill}}}{
			\par The candidate must successfully complete an individual project. The project is assessed using a number of reports, oral presentations and poster as stipulated in the due dates section at the start of this document. The candidate:
			\begin{enumerate}
				\item Brings mathematical and numerical analysis to bear on engineering problems by using an appropriate mix of:
				      \begin{enumerate}
					      \item Formal analysis and modelling of engineering components, systems or processes
					      \item Communicating concepts, ideas and theories with the aid of mathematics
					      \item Reasoning about and conceptualising engineering components, systems or processes using mathematical concepts
				      \end{enumerate}
				\item Uses physical laws and knowledge of the physical world as a foundation for the engineering sciences and the solution of engineering problems by an appropriate mix of:
				      \begin{enumerate}
					      \item Formal analysis and modelling of engineering components, systems or processes using principles and knowledge of the basic sciences
					      \item Reasoning about and conceptualising engineering problems, components, systems or processes using principles of the basic sciences
				      \end{enumerate}
				\item Uses the techniques, principles and laws of engineering science at a fundamental level and in at least one specialist area to:
				      \begin{enumerate}
					      \item Identify and solve open-ended engineering problems
					      \item Identify and pursue engineering applications
					      \item Work across engineering disciplinary boundaries through cross disciplinary literacy and shared fundamental knowledge
				      \end{enumerate}
			\end{enumerate}
			\par}                                                                                                                                                                                                                            \\
			\hline
			\multicolumn{2}{|>{\small\sffamily\bfseries\columncolor[gray]{.8}}c|}{\capitalisewords{ELO 3: Engineering Design}}                                                                                                               \\
			\nobreakhline
			Demonstrate competence to perform creative, procedural and non-procedural design and synthesis of components, systems, engineering works, products or processes. & \textbullet \space Section X                                  \\
			\nobreakhline
			\multicolumn{2}
			{@{\hspace{\fill}} >{\small\normalfont\justifying}p{\textwidth} @{\hspace{\fill}}}{
			\par The candidate designs components, systems, engineering works, products or processes as part of the project. The design process and its outcome is documented in the report. The candidate executes an acceptable design process encompassing the following:
			\begin{enumerate}
				\item Plans and manages the design process: focuses on important issues, recognises and deals with constraints
				\item Acquires and evaluates the requisite knowledge, information and resources: applies correct principles, evaluates and uses design tools
				\item Performs design tasks including analysis, quantitative modelling and optimisation
				\item Evaluates alternatives and preferred solution: exercises judgement, tests implement ability and performs techno-economic analyses
				\item Communicates the design logic and information
			\end{enumerate}
			\par}                                                                                                                                                                                                                            \\
			\hline
			\multicolumn{2}{|>{\small\sffamily\bfseries\columncolor[gray]{.8}}c|}{\capitalisewords{ELO 5: Engineering methods, skills and tools, including Information Technology}}                                                          \\
			\nobreakhline
			Demonstrate competence to use appropriate engineering methods, skills and tools, including those based on information technology.                                & \textbullet \space Section X                                  \\
			\nobreakhline
			\multicolumn{2}
			{@{\hspace{\fill}} >{\small\normalfont\justifying}p{\textwidth} @{\hspace{\fill}}}{
			\par Sufficient demonstration of the critical use of applicable engineering methods, skills and tools at the level of 3rd or 4th year BEng is required. The candidate:
			\begin{enumerate}
				\item Uses method, skill or tool effectively by:
				      \begin{enumerate}
					      \item Selecting and assessing the applicability and limitations of the method, skill or tool
					      \item Properly applying the method, skill or tool
					      \item Critically testing and assessing the end-results produced by the method, skill or tool
				      \end{enumerate}
				\item Creates computer applications as required by the discipline
			\end{enumerate}
			\par}                                                                                                                                                                                                                            \\
			\hline
			\multicolumn{2}{|>{\small\sffamily\bfseries\columncolor[gray]{.8}}c|}{\capitalisewords{ELO 6: Professional and technical communication}}                                                                                         \\
			\nobreakhline
			Demonstrate competence to communicate effectively, both orally and in writing, with engineering audiences and the community at large.                            & \textbullet \space Section X                                  \\
			\nobreakhline
			\multicolumn{2}
			{@{\hspace{\fill}} >{\small\normalfont\justifying}p{\textwidth} @{\hspace{\fill}}}{
			\par The candidate demonstrated: The communication was clear and understandable; Oral presentations, poster and final report are professionally acceptable; Language usage is as required for technical communication. The candidate executes effective written communication as evidenced by:
			\begin{enumerate}
				\item Uses appropriate structure, style and language for purpose and audience
				\item Uses effective graphical support
				\item Applies methods of providing information for use by others involved in engineering activity
				\item Meets the requirements of the target audience
			\end{enumerate}
			The candidate executes effective oral communication as evidenced by:
			\begin{enumerate}
				\item Uses appropriate structure, style and language
				\item Uses appropriate visual materials
				\item Delivers fluently
				\item Meets the requirements of the intended audience
			\end{enumerate}
			\par}                                                                                                                                                                                                                            \\
			\hline
			\multicolumn{2}{|>{\small\sffamily\bfseries\columncolor[gray]{.8}}c|}{\capitalisewords{ELO 8: Individual, Team and Multidisciplinary Working}}                                                                                   \\
			\nobreakhline
			Demonstrate competence to work effectively as an individual, in teams and in multi-disciplinary environments.                                                    & \textbullet \space Section X                                  \\
			\nobreakhline
			\multicolumn{2}
			{@{\hspace{\fill}} >{\small\normalfont\justifying}p{\textwidth} @{\hspace{\fill}}}{
			\par The candidate demonstrated: The main objectives of the project were achieved; The student's work was focussed on the objectives and well planned; Moderate supervision was required. The candidate demonstrates effective individual work by performing the following:
			\begin{enumerate}
				\item Identifies and focuses on objectives
				\item Works strategically
				\item Executes tasks effectively
				\item Delivers completed work on time
			\end{enumerate}
			\par}                                                                                                                                                                                                                            \\
			\hline
			\multicolumn{2}{|>{\small\sffamily\bfseries\columncolor[gray]{.8}}c|}{\capitalisewords{ELO 9: Independent Learning Ability}}                                                                                                     \\
			\nobreakhline
			Demonstrate competence to engage in independent learning through well-developed learning skills.                                                                 & \textbullet \space Section X                                  \\
			\nobreakhline
			\multicolumn{2}
			{@{\hspace{\fill}} >{\small\normalfont\justifying}p{\textwidth} @{\hspace{\fill}}}{
			\par The candidate demonstrated: Applicable independent research was conducted and sensibly used in the project; The source material used was at the level of a 3rd or 4th year text book. The candidate shows evidence of being an effective independent learner by the following:
			\begin{enumerate}
				\item Reflects on own learning and determines learning requirements and strategies
				\item Sources and evaluates information
				\item Accesses, comprehends and applies knowledge acquired outside formal instruction
				\item Critically challenges assumptions and embraces new thinking
			\end{enumerate}
			\par}
		\end{longtable}
	\end{USS@SetMargins}
	\clearpage
}
\makeatother


