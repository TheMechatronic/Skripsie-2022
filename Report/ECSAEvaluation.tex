% Title: Report LaTex File: ECSA Evaluation - Filled in
% Auther: DC Eksteen
% Student Number: 22623906
% Contact: 22623906@sun.ac.za
% Date: 2022/09/14
% Version: 2.0

\makeatletter
\newcommand{\ECSApage}{
	\clearpage
	\vspace*{15mm - 1in-\voffset-\topmargin-\headheight-\headsep-\topskip}
	\thispagestyle{plain}%
	\centering
	\phantomsection
	\addcontentsline{toc}{chapter}{ECSA Exit Level Outcome Evaluation}%
	\begin{USS@SetMargins}{25mm}{25mm}
		\centering\large\sffamily\bfseries\MakeUppercase{\DegreeNameLong: \\ECSA Exit Level Outcome Evaluation}
		\bigskip
		\setlength\LTleft{\leftmargin}%
		\setlength\LTright{\rightmargin}
		\renewcommand{\arraystretch}{1}
		\small
		\centering
		\begin{longtable}{@{\extracolsep{\fill}}| >{\raggedright}p{.85\textwidth} | >{\raggedright\noindent\arraybackslash}p{32mm} |}%
			\hline
			\multicolumn{2}{|>{\small\sffamily\bfseries\columncolor[gray]{.8}}c|}{\capitalisewords{ELO 1: Problem Solving}}                                                          \\
			\hline
			Demonstrate competence to identify, assess, formulate and solve convergent and divergent engineering problems creatively and innovatively.                       & \textbullet \space Chapters 1,2,3 and 4\newline \textbullet\space Poster \\*
			\nobreakhline
			\multicolumn{2}
			{@{\hspace{\fill}} >{\small\normalfont\justifying}p{\textwidth} @{\hspace{\fill}}}{
				\begin{itemize}[leftmargin=*]
					\item Identify the need for an inexpensive smart trainer that will be able to interface with Zwift.
					\item Indicate the criteria needed to connect with Zwift, as well as have a usable trainer done in Introcuction and Literature Review. (Chapter 1 and 2).
					\item Best sollution for general problem presented, then split into seperate sections to solve further ito. Software and Hardware requirements (Chapter 3 and 4)
				\end{itemize}}\\
			\hline
			\multicolumn{2}{|>{\small\sffamily\bfseries\columncolor[gray]{.8}}c|}{\capitalisewords{ELO 2: Application of scientific and engineering knowledge}}                      \\
			\nobreakhline
			Demonstrate competence to apply knowledge of mathematics, basic science and engineering sciences from first principles to solve engineering problems.            & \textbullet \space Section 4.1 \\*
			\nobreakhline
			\multicolumn{2}
			{@{\hspace{\fill}} >{\small\normalfont\justifying}p{\textwidth} @{\hspace{\fill}}}{
				\begin{itemize}[leftmargin=*]
					\item Model of Forces involved in trainer for use in basic hardware design and component selection.
					\item Mathematical model and analysis of Eddy Current Brake. (Section 4.1)
				\end{itemize}
			}\\
			\hline
			\multicolumn{2}{|>{\small\sffamily\bfseries\columncolor[gray]{.8}}c|}{\capitalisewords{
					ELO 3: Engineering Design}}                                                       \\
			\nobreakhline
			Demonstrate competence to perform creative, procedural and non-procedural design and synthesis of components, systems, engineering works, products or processes. & \textbullet \space Chapter 2 and 3 \\
			\nobreakhline
			\multicolumn{2}
			{@{\hspace{\fill}} >{\small\normalfont\justifying}p{\textwidth} @{\hspace{\fill}}}{
				\begin{itemize}[leftmargin=*]
					\item Design the software approach, communicating with Zwift through BLE protocol.
					\item Design a hardware approach for both the resistance brake and the general trainer.
					\item Design the electro-mechanical control of the braking unit and the controlling software. (Chapter 2 and 3)
				\end{itemize}
			}\\
			\hline
			\multicolumn{2}{|>{\small\sffamily\bfseries\columncolor[gray]{.8}}c|}{\capitalisewords{ELO 5: Engineering methods, skills and tools, including Information Technology}}  \\
			\nobreakhline
			Demonstrate competence to use appropriate engineering methods, skills and tools, including those based on information technology.                                & \textbullet \space Chapter 3 and 4 \\
			\nobreakhline
			\multicolumn{2}
			{@{\hspace{\fill}} >{\small\normalfont\justifying}p{\textwidth} @{\hspace{\fill}}}{
				\begin{itemize}[leftmargin=*]
			 		\item Engineering method applied to solving the communication and control of the trainer with an external Zwift host.
					\item Engineering method applied to selecting, designing and implementing the hardware components needed for the trainer.
					\item Adequate consideration for the goal of the product applied to the planning and design phases.
				\end{itemize}
			}\\
			\hline
			\multicolumn{2}{|>{\small\sffamily\bfseries\columncolor[gray]{.8}}c|}{\capitalisewords{ELO 6: Professional and technical communication}}                                 \\
			\nobreakhline
			Demonstrate competence to communicate effectively, both orally and in writing, with engineering audiences and the community at large.                            & \textbullet \space Report \newline \textbullet \space Poster \newline \textbullet \space Presentation \\
			\nobreakhline
			\multicolumn{2}
			{@{\hspace{\fill}} >{\small\normalfont\justifying}p{\textwidth} @{\hspace{\fill}}}{
				\begin{itemize}[leftmargin=*]
					\item The Report and the oral presentation will aim to prove the ability to achieve the communication.
					\item The Report will focus on presenting the project to a technical and engineering audience where the poster will be aimed at presenting to the wider community.
				\end{itemize}
			}\\
			\hline
			\multicolumn{2}{|>{\small\sffamily\bfseries\columncolor[gray]{.8}}c|}{\capitalisewords{ELO 8: Individual, Team and Multidisciplinary Working}}                           \\
			\nobreakhline
			Demonstrate competence to work effectively as an individual, in teams and in multi-disciplinary environments.                                                    & \textbullet \space Section X \\
			\nobreakhline
			\multicolumn{2}
			{@{\hspace{\fill}} >{\small\normalfont\justifying}p{\textwidth} @{\hspace{\fill}}}{
				\begin{itemize}[leftmargin=*]
					\item Project focuses on Individual work, with guidance from supervisor and input from other students working under same supervisor.
					\item The project proposal, progress report, progress presentation, draft report and final report will be handed in to achieve this.
				\end{itemize}
			}\\
			\hline
			\multicolumn{2}{|>{\small\sffamily\bfseries\columncolor[gray]{.8}}c|}{\capitalisewords{ELO 9: Independent Learning Ability}}                                             \\
			\nobreakhline
			Demonstrate competence to engage in independent learning through well-developed learning skills.                                                                 
			& \textbullet \space Chapter 2 \& 3 \\
			\nobreakhline
			\multicolumn{2}
			{@{\hspace{\fill}} >{\small\normalfont\justifying}p{\textwidth} @{\hspace{\fill}}}{
				\begin{itemize}[leftmargin=*]
					\item The learning of Bluetooth technology and application of Eddy Current Brake models prove independent learning.
				\end{itemize}
			}
		\end{longtable}
	\end{USS@SetMargins}
	\clearpage
}
\makeatother


\makeatletter
\newcommand{\ECSApageInfo}{
	\clearpage
	\vspace*{15mm - 1in-\voffset-\topmargin-\headheight-\headsep-\topskip}
	\thispagestyle{plain}%
	\centering
	\phantomsection
	\addcontentsline{toc}{chapter}{ECSA Exit Level Outcome Evaluation}%
	\begin{USS@SetMargins}{25mm}{25mm}
		\centering\large\sffamily\bfseries\MakeUppercase{\DegreeNameLong: \\ECSA Exit Level Outcome Evaluation}
		\bigskip
		\setlength\LTleft{\leftmargin}%
		\setlength\LTright{\rightmargin}
		\renewcommand{\arraystretch}{1}
		\small
		\centering
		\begin{longtable}{@{\extracolsep{\fill}}| >{\raggedright}p{.85\textwidth} | >{\raggedright\noindent\arraybackslash}p{32mm} |}%
			\hline
			\multicolumn{2}{|>{\small\sffamily\bfseries\columncolor[gray]{.8}}c|}{\capitalisewords{ELO 1: Problem Solving}}                                                          \\
			\hline
			Demonstrate competence to identify, assess, formulate and solve convergent and divergent engineering problems creatively and innovatively.                       & \textbullet \space Section 1\newline \textbullet\space Poster \\*
			\nobreakhline
			\multicolumn{2}
			{@{\hspace{\fill}} >{\small\normalfont\justifying}p{\textwidth} @{\hspace{\fill}}}{\par The candidate must successfully complete an individual project. The project is
				assessed using a number of reports, oral presentations and poster as stipulated in the
				due dates section at the start of this document.
				The candidate applies in a number of varied instances, a systematic problem solving
				method including:
				\begin{enumerate}
					\item Analyses and defines the problem, identifies the criteria for an acceptable solution
					\item Identifies necessary information and applicable engineering and other knowledge and skills
					\item Generates and formulates possible approaches to solution of problem
					\item Models and analyses possible solution(s)
					\item Evaluates possible solutions and selects best solution
					\item Formulates and presents the solution in an appropriate form.
			\end{enumerate}}                                                                          \\
			\hline
			\multicolumn{2}{|>{\small\sffamily\bfseries\columncolor[gray]{.8}}c|}{\capitalisewords{ELO 2: Application of scientific and engineering knowledge}}                      \\
			\nobreakhline
			Demonstrate competence to apply knowledge of mathematics, basic science and engineering sciences from first principles to solve engineering problems.            & \textbullet \space Section X \\*
			\nobreakhline
			\multicolumn{2}
			{@{\hspace{\fill}} >{\small\normalfont\justifying}p{\textwidth} @{\hspace{\fill}}}{
				\par The candidate must successfully complete an individual project. The project is assessed using a number of reports, oral presentations and poster as stipulated in the due dates section at the start of this document. The candidate:
				\begin{enumerate}
					\item Brings mathematical and numerical analysis to bear on engineering problems by using an appropriate mix of:
					\begin{enumerate}
						\item Formal analysis and modelling of engineering components, systems or processes
						\item Communicating concepts, ideas and theories with the aid of mathematics
						\item Reasoning about and conceptualising engineering components, systems or processes using mathematical concepts
					\end{enumerate}
					\item Uses physical laws and knowledge of the physical world as a foundation for the engineering sciences and the solution of engineering problems by an appropriate mix of:
					\begin{enumerate}
						\item Formal analysis and modelling of engineering components, systems or processes using principles and knowledge of the basic sciences
						\item Reasoning about and conceptualising engineering problems, components, systems or processes using principles of the basic sciences
					\end{enumerate}
					\item Uses the techniques, principles and laws of engineering science at a fundamental level and in at least one specialist area to:
					\begin{enumerate}
						\item Identify and solve open-ended engineering problems
						\item Identify and pursue engineering applications
						\item Work across engineering disciplinary boundaries through cross disciplinary literacy and shared fundamental knowledge
					\end{enumerate}
				\end{enumerate}
				\par}                                                                                                                                                                    \\
			\hline
			\multicolumn{2}{|>{\small\sffamily\bfseries\columncolor[gray]{.8}}c|}{\capitalisewords{ELO 3: Engineering Design}}                                                       \\
			\nobreakhline
			Demonstrate competence to perform creative, procedural and non-procedural design and synthesis of components, systems, engineering works, products or processes. & \textbullet \space Section X \\
			\nobreakhline
			\multicolumn{2}
			{@{\hspace{\fill}} >{\small\normalfont\justifying}p{\textwidth} @{\hspace{\fill}}}{
				\par The candidate designs components, systems, engineering works, products or processes as part of the project. The design process and its outcome is documented in the report. The candidate executes an acceptable design process encompassing the following:
				\begin{enumerate}
					\item Plans and manages the design process: focuses on important issues, recognises and deals with constraints
					\item Acquires and evaluates the requisite knowledge, information and resources: applies correct principles, evaluates and uses design tools
					\item Performs design tasks including analysis, quantitative modelling and optimisation
					\item Evaluates alternatives and preferred solution: exercises judgement, tests implement ability and performs techno-economic analyses
					\item Communicates the design logic and information
				\end{enumerate}
				\par}                                                                                                                                                                    \\
			\hline
			\multicolumn{2}{|>{\small\sffamily\bfseries\columncolor[gray]{.8}}c|}{\capitalisewords{ELO 5: Engineering methods, skills and tools, including Information Technology}}  \\
			\nobreakhline
			Demonstrate competence to use appropriate engineering methods, skills and tools, including those based on information technology.                                & \textbullet \space Section X \\
			\nobreakhline
			\multicolumn{2}
			{@{\hspace{\fill}} >{\small\normalfont\justifying}p{\textwidth} @{\hspace{\fill}}}{
				\par Sufficient demonstration of the critical use of applicable engineering methods, skills and tools at the level of 3rd or 4th year BEng is required. The candidate:
				\begin{enumerate}
					\item Uses method, skill or tool effectively by:
					\begin{enumerate}
						\item Selecting and assessing the applicability and limitations of the method, skill or tool
						\item Properly applying the method, skill or tool
						\item Critically testing and assessing the end-results produced by the method, skill or tool
					\end{enumerate}
					\item Creates computer applications as required by the discipline
				\end{enumerate}
				\par}                                                                                                                                                                    \\
			\hline
			\multicolumn{2}{|>{\small\sffamily\bfseries\columncolor[gray]{.8}}c|}{\capitalisewords{ELO 6: Professional and technical communication}}                                 \\
			\nobreakhline
			Demonstrate competence to communicate effectively, both orally and in writing, with engineering audiences and the community at large.                            & \textbullet \space Section X \\
			\nobreakhline
			\multicolumn{2}
			{@{\hspace{\fill}} >{\small\normalfont\justifying}p{\textwidth} @{\hspace{\fill}}}{
				\par The candidate demonstrated: The communication was clear and understandable; Oral presentations, poster and final report are professionally acceptable; Language usage is as required for technical communication. The candidate executes effective written communication as evidenced by:
				\begin{enumerate}
					\item Uses appropriate structure, style and language for purpose and audience
					\item Uses effective graphical support
					\item Applies methods of providing information for use by others involved in engineering activity
					\item Meets the requirements of the target audience
				\end{enumerate}
				The candidate executes effective oral communication as evidenced by:
				\begin{enumerate}
					\item Uses appropriate structure, style and language
					\item Uses appropriate visual materials
					\item Delivers fluently
					\item Meets the requirements of the intended audience
				\end{enumerate}
				\par}                                                                                                                                                                    \\
			\hline
			\multicolumn{2}{|>{\small\sffamily\bfseries\columncolor[gray]{.8}}c|}{\capitalisewords{ELO 8: Individual, Team and Multidisciplinary Working}}                           \\
			\nobreakhline
			Demonstrate competence to work effectively as an individual, in teams and in multi-disciplinary environments.                                                    & \textbullet \space Section X \\
			\nobreakhline
			\multicolumn{2}
			{@{\hspace{\fill}} >{\small\normalfont\justifying}p{\textwidth} @{\hspace{\fill}}}{
				\par The candidate demonstrated: The main objectives of the project were achieved; The student's work was focussed on the objectives and well planned; Moderate supervision was required. The candidate demonstrates effective individual work by performing the following:
				\begin{enumerate}
					\item Identifies and focuses on objectives
					\item Works strategically
					\item Executes tasks effectively
					\item Delivers completed work on time
				\end{enumerate}
				\par}                                                                                                                                                                    \\
			\hline
			\multicolumn{2}{|>{\small\sffamily\bfseries\columncolor[gray]{.8}}c|}{\capitalisewords{ELO 9: Independent Learning Ability}}                                             \\
			\nobreakhline
			Demonstrate competence to engage in independent learning through well-developed learning skills.                                                                 & \textbullet \space Section X \\
			\nobreakhline
			\multicolumn{2}
			{@{\hspace{\fill}} >{\small\normalfont\justifying}p{\textwidth} @{\hspace{\fill}}}{
				\par The candidate demonstrated: Applicable independent research was conducted and sensibly used in the project; The source material used was at the level of a 3rd or 4th year text book. The candidate shows evidence of being an effective independent learner by the following:
				\begin{enumerate}
					\item Reflects on own learning and determines learning requirements and strategies
					\item Sources and evaluates information
					\item Accesses, comprehends and applies knowledge acquired outside formal instruction
					\item Critically challenges assumptions and embraces new thinking
				\end{enumerate}
				\par}
		\end{longtable}
	\end{USS@SetMargins}
	\clearpage
}
\makeatother


