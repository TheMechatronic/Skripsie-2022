\chapter{Conclusion}
\label{ch:conclusion}

This project set out to investigate the requirements of developing a smart trainer that is capable of interacting with the Zwift application, as a barrier for many consumers was identified. These requirements were then developed into a conceptual solution, where the necessary software will be made available under an open-source licence along with a demonstration of the development process required to create a smart trainer utilising the developed software.

The hardware objectives that the project set out to achieve were to develop an inexpensive bicycle training platform that is capable of both monitoring and adjusting the training experience while interacting with Zwift. This was achieved by demonstrating the development of a roller trainer with an adjustable eddy current brake and speed sensors on the rollers.

The software objectives that the project set out to achieve was to collect and process training conditions and inputs, communicating with Zwift and integrating the Zwift experience with the developed hardware. This was achieved by developing an application running on a Raspberry Pi that demonstrates the ability to fully interact with the Zwift application.

Finally the final deliverable objectives that the project aimed to demonstrate was a smart training platform that would be accessible to a wide range of consumers, while also having comparable features to similar trainers available on the market. \\

The project also aimed to more broadly provide a solution framework and supporting code that will enable future users and developers to create more examples of self-made smart trainers. This was all achieved during the progress of this project by releasing all information that would be required to utilise the developed framework to a public GitHub repository.

\subsubsection{Recommendation for Future Work}

Although the project does not allow for much further development with research value, further demonstrations of developed and implemented trainers would contribute to the general pool of knowledge about the development of self-made smart bicycle trainers.

Should further work be done utilising the software developed in this project, changing the application to rather utilise C++ code would allow the application to be implemented on other micro-controllers such as Arduino or ESP32 boards.