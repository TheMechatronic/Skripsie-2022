% Title: Report LaTex File: Introduction
% Auther: DC Eksteen
% Student Number: 22623906
% Contact: 22623906@sun.ac.za
% Date: 2022/09/19
% Version: 2.2

\chapter{Introduction}

\section{Background}
% Background on the topic - Try to keep to single page
% Opening statement
% Why cycling as focus area:
Bad weather, time, safety or other factors may prevent cyclists from being able to train outside.
% Why trainers are needed:
Modern city lifestyle often means that it is often safer and more convenient 
In recent years, it has become very popular for cyclists to train indoors as this is often safer than cycling on busy city streets, it makes cycling less reliant on the weather and is more convenient. The problem is that indoor cycling can often be very boring and lonely. \\
% What is Zwift:
Zwift is an online based virtual reality platform that allows indoor cyclists to virtually ride courses in the real world whilst communicating and competing with other people in real time through the platform. This allows for indoor cycling to become more interactive and exciting, allowing the cyclist to train more effectively and enabling social engagement during training sessions.
% Why is project needed
In order to use the Zwift platform, you need a compatible trainer or training device in order to send riding information to Zwift and to receive feedback that will adjust the riding experience based on the virtual circumstances. These trainers are very expensive and thus not very accessible to many consumers.
% Final closing statement
This project will aim to demonstrate the development, design and testing of an inexpensive smart bicycle trainer. The project, conducted by Mr D.C. Eksteen as part of Mechatronic Project 475, stems from a proposal by Dr G. Venter.

\section{Aim and Objectives}

This project aims to develop, build and test a smart trainer, with specific focus on compatibility with Zwift.\\
The objectives will be to: 

\begin{itemize}
	\item Develop Hardware to:
	\begin{enumerate}
		\item Monitor and measure the rider inputs. These include, for example, cycling speed, cadence, power, heart rate etc.
		\item Change the riding conditions based on feedback from Zwift platform. These include, for example, rolling resistance, incline etc.
	\end{enumerate}
	
	\item Develop Software to:
	\begin{enumerate}
		\item Collect and process rider inputs and to communicate this to the Zwift platform.
		\item Receive feedback information from Zwift and enable hardware to change the riding conditions as mentioned above.
	\end{enumerate}
	\item Combine developed Hardware and Software to produce a smart trainer that is comparable to smart trainers that are available on the market, but at a lower cost point. (Needs refinement)
\end{itemize} % List the hardware and software requirements

\newpage

\section{Motivation}
% This is very important -- refer to notes
% What
% Why
% How

Motivation will focus on:
\begin{itemize}
	\item What:
	\begin{itemize}
		\item Zwift = popular indoor training and competitive platform.
		\item Bicycle trainer = Convenient way to train cycling without needing to cycle outside.
	\end{itemize}
	\item Why:
	\begin{itemize}
		\item Indoor training increasing in popularity.
		\item Making access to Zwift more accessible and thus promoting cycling and exercise. 
	\end{itemize}
	\item How:
	\begin{itemize}
		\item Convert existing "dumb" trainer into smart trainer.
		\item Enable trainer to connect with Zwift platform.
		\item Using and leveraging existing industry standards to produce a robust solution.
	\end{itemize}
	
\end{itemize}